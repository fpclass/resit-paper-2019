%-----------------------------------------------------------
% LaTeX template for University of Warwick exams
%-----------------------------------------------------------

\usepackage{fancyeq}
\usepackage{tikz}
\usetikzlibrary{automata, positioning, arrows}
\makeatletter
\expandafter\providecommand\expandafter*\csname ver@framed.sty\endcsname
{2003/07/21 v0.8a Simulated by exam}
\makeatother
\usepackage{framed}
\usepackage{xcolor}
\usepackage{minted}
\usepackage[
backgroundcolor = white
, hidealllines=true
]{mdframed}

\surroundwithmdframed{minted}
\usemintedstyle{bw}

\setminted[text]{fontsize=\small}
\setminted[haskell]{fontsize=\small}
\setminted[bash]{fontsize=\small}
\newcommand{\haskellIn}[1]{\mintinline[fontsize=\small]{text}{#1}}
\newcommand{\bashIn}[1]{\mintinline[fontsize=\small]{bash}{#1}}

\usepackage[]{FiraSans}

\usepackage{mathpazo}
\linespread{1.05}         % Palatino needs more leading (space between lines)
\usepackage[T1]{fontenc}

\usepackage{multicol}

%\usepackage[nomap]{FiraMono}

% configure the heading 
\ModuleName{Functional Programming}
\ExamPeriod{September 2019}
\TimeAllowed{2 hours}
\QuestionInstructions{Answer \textbf{FOUR} questions.}
\OtherInstructions{Read carefully the instructions on the answer book and make sure that the particulars required are entered on \textbf{each} answer book. \\

No calculators are allowed. A copy of the Haskell Prelude is given in the Appendix.}

\usepackage[nomap]{FiraMono}

\usepackage{microtype}
\DisableLigatures[f]{encoding = *, family = tt* }

\begin{document}
	\MakeHeading
	
	\begin{questions}
		%%% Question 1 - - - - - - - - - - - - - - - - - - - - - - - - - - - - - -
\question This question is about functional programming as a programming paradigm.

\makeatletter
\renewcommand{\verbatim@font}{\ttfamily \small}
\makeatother

\begin{parts}
	
	\part For each of the following Haskell expressions, reduce it to its normal form.
	
	\begin{subparts}
		
		\subpart[1] \haskellIn{last "jedi"}  \droppoints
		
		\begin{solution}
			\emph{Comprehension.} \haskellIn{'i'}
		\end{solution}
		
		\subpart[1] \haskellIn{reverse (filter (>'g') "uaogyfsetdichboa")} \droppoints 
		
		\begin{solution}
		\emph{Comprehension.} \haskellIn{"ohitsyou"} 
		\end{solution}
	
		\subpart[1] \haskellIn{fmap (fmap (\f a -> (*2) . f a)) (Just [(+),(-)])}  \droppoints
		
		\begin{solution}
		\emph{Comprehension.}\\ \haskellIn{Just [\a -> (*2) . (+) a, \a -> (*2) . (-) a]}
		\end{solution}
	
		\subpart[1] {\small \texttt{foldr}\\
		\texttt{\phantom{~~}(\textbackslash x r -> foldr take r x ++ r)}\\
		\texttt{\phantom{~~}"NaBatman" [Just 4, Just 2]}}  \droppoints
		
		\begin{solution}
			\emph{Comprehension.}\\ \haskellIn{"NaNaNaNaBatman"}
		\end{solution}
	
		\subpart[1] 
		\begin{verbatim}
		let a = Just 4
		    b = Just 5 
		in (pure (+) <*> a <*> b) ==
		\end{verbatim}
		\texttt{\small \phantom{~~~}(a >{}>= \textbackslash x -> b >{}>= \textbackslash y -> pure (x+y))}
		\droppoints
		
		\begin{solution}
		\emph{Comprehension.} \haskellIn{True}
		\end{solution}
	\end{subparts}
	
	\part For each of the following expressions, choose the permissible type from the options that are listed if there is one. There is \emph{at most} one correct option for each expression. If no option is correct, please write ``not well typed''.
	
	\begin{subparts}
		
    	\subpart[1] \haskellIn{drop 4} \droppoints
    	
	    \begin{enumerate}
	    	\item \haskellIn{Int -> [a] -> [a]}
	    	\item \haskellIn{[a] -> [a]}
	    	\item \haskellIn{[a]}
	    	\item \haskellIn{Num a => [a]}
	    	\item \haskellIn{Num a => (Int -> [a] -> [a]) -> a}
	    \end{enumerate}
    
		\begin{solution}
			\emph{Comprehension.} 2
		\end{solution}
	
    	\subpart[1] \haskellIn{filter True [15==16,23==42]} \droppoints
    	
	    \begin{enumerate}
	    	\item \haskellIn{Bool -> [Bool] -> [Bool]}
	    	\item \haskellIn{Bool -> [a] -> [a]}
	    	\item \haskellIn{Bool -> [Int -> Int -> Bool] -> [Bool]}
	    	\item \haskellIn{[Bool]}
	    	\item \haskellIn{[Int -> Int -> Bool]}
	    \end{enumerate}
    
	    \begin{solution}
	   		\emph{Comprehension.} Not well typed.
	    \end{solution}
    
    	%\ifprintanswers \pagebreak \else \fi
    
    	%\ifprintanswers \else \pagebreak \fi
    	
    	\subpart[1] \haskellIn{let i=8+i in i} \droppoints
    	
	    \begin{enumerate}
	    	\item \haskellIn{a}
	    	\item \haskellIn{Num a => a}
	    	\item \haskellIn{a -> a}
	    	\item \haskellIn{Num a => a -> a}
	    	\item \haskellIn{Num a => (a,a)}
	    \end{enumerate}
    
	    \begin{solution}
	    	\emph{Comprehension.} 2
	    \end{solution}
    
		\subpart[1] \haskellIn{fmap . fmap} \droppoints
		
		\begin{enumerate}
			\item \haskellIn{Functor f => (f a -> f b) -> f a -> f b}
			\item \haskellIn{(Functor f, Functor g) => (a -> b) -> f (g a) -> f (g b)}
			\item \haskellIn{(Functor f, Functor g) => (f a -> g b) -> f a -> f b}
			\item \haskellIn{Functor f => f a -> f b}
			\item \haskellIn{(Functor f, Functor g) => f a -> g b}
		\end{enumerate}
	
		\begin{solution}
			\emph{Comprehension.} 2
		\end{solution}
	
		\ifprintanswers \else \pagebreak \fi
		
    	\subpart[1] \haskellIn{\f -> (\x -> f (x x)) (\x -> f (x x))} \droppoints 
    	
	    \begin{enumerate}
	    	\item \haskellIn{a -> b}
	    	\item \haskellIn{(a -> b) -> b}
	    	\item \haskellIn{(a -> b) -> (b -> b) -> b}
	    	\item \haskellIn{(a -> (b -> b)) -> b}
	    	\item \haskellIn{(a -> (b -> b)) -> (b -> b) -> b}
	    \end{enumerate}
    
	    \begin{solution}
	    	\emph{Comprehension.} Not well typed.
	    \end{solution}
	\end{subparts}

	%\ifprintanswers \else \pagebreak \fi

	\part[5] Consider the following definition of \haskellIn{map}:
	\begin{small}
	\begin{verbatim}
	map :: (a -> b) -> [a] -> [b]
	map f []     = []
	map f (x:xs) = f x : map f xs
	\end{verbatim}
	\end{small}
	
	Define a function \haskellIn{map'} which is equivalent to \haskellIn{map}, but is defined using \haskellIn{foldl} instead of explicit recursion. \droppoints 
	
	\begin{solution}
	\emph{Application.} One possible answer is:
	\begin{verbatim}
	map' f = foldl (\r x -> r ++ [f x]) []
	\end{verbatim}
	\end{solution}
\part  \label{part:sum}
\begin{subparts}
\subpart[5] \label{part:strict} Trace how \haskellIn{map' (+1) [4,8,15]} would be evaluated in a language with \emph{call-by-value} semantics. \droppoints
\begin{solution}
\emph{Comprehension.}
\begin{small}
\begin{verbatim}
   map' (+1) [4,8,15]
=> foldl (\r x -> r ++ [(+1) x]) [] [4,8,15]
=> foldl (\r x -> r ++ [(+1) x]) ([] ++ [(+1) 4]) [8,15]
=> foldl (\r x -> r ++ [(+1) x]) ([] ++ [5]) [8,15]
=> foldl (\r x -> r ++ [(+1) x]) [5] [8,15]
=> foldl (\r x -> r ++ [(+1) x]) ([5] ++ [(+1) 8]) [15]
=> foldl (\r x -> r ++ [(+1) x]) ([5] ++ [9]) [15]
=> foldl (\r x -> r ++ [(+1) x]) [5,9] [15]
=> foldl (\r x -> r ++ [(+1) x]) ([5,9] ++ [(+1) 15]) []
=> foldl (\r x -> r ++ [(+1) x]) ([5,9] ++ [16]) []
=> foldl (\r x -> r ++ [(+1) x]) [5,9,16] []
=> [5,9,16]
\end{verbatim}
\end{small}
\end{solution}
\subpart[5] \label{part:lazy} Trace how \haskellIn{map' (+1) [4,8,15]} would be evaluated in a language with \emph{call-by-name} semantics. You should assume that the value of this expression is required by some other part of the program. \droppoints
\begin{solution} \emph{Comprehension.}
\begin{small}
\begin{verbatim}
   map' (+1) [4,8,15]
=> foldl (\r x -> r ++ [(+1) x]) [] [4,8,15]
=> foldl (\r x -> r ++ [(+1) x]) ([] ++ [(+1) 4]) [8,15]
=> foldl (\r x -> r ++ [(+1) x]) (([] ++ [(+1) 4]) 
   ++ [(+1) 8]) [15]
=> foldl (\r x -> r ++ [(+1) x]) ((([] ++ [(+1) 4]) 
   ++ [(+1) 8]) ++ [(+1) 15]) []
=> (([] ++ [(+1) 4]) ++ [(+1) 8]) ++ [(+1) 15]
=> ([(+1) 4] ++ [(+1) 8]) ++ [(+1) 15]
=> [(+1) 4, (+1) 8] ++ [(+1) 15]
=> [(+1) 4, (+1) 8, (+1) 15]
=> [5,9,16]
\end{verbatim}
\end{small}
\end{solution}
\end{subparts}
\end{parts}
 % done
		%%% Question 2 - - - - - - - - - - - - - - - - - - - - - - - - - - - - - -
\question This question is about recursive and higher-order functions.


In the \emph{countdown problem} you are given a list of numbers, the standard arithmetic operators (\texttt{\small +}, \texttt{\small -}, \texttt{\small *}, \texttt{\small /}), and a target number. The goal is to construct an arithmetic expression using the given list of numbers and the standard arithmetic operators which evaluates to the target number. Each number may only be used once and values, including intermediate results, can never be negative. For example, if the list of numbers is \texttt{\small [1, 3, 7, 10, 25, 50]} and the target is \texttt{\small 765}, then a valid solution to the countdown problem is \texttt{\small (1 + 50) * (25 - 10)}. We represent operators and expressions as follows:
\begin{verbatim}
data Op   = Add | Sub | Mul | Div
data Expr = Val Int | App Op Expr Expr
\end{verbatim}
\begin{parts}
	\part[4] Define a function 
	\begin{center}
		\texttt{\small split~::~[a] -> [([a],[a])]}
	\end{center}
	which returns a list of pairs describing all possible ways to split a list into two non-empty ones. For example, \texttt{\small split [1,2,3,4]} should evaluate to \texttt{\small [([1],[2,3,4]), ([1,2],[3,4]), ([1,2,3],[4])]}. \droppoints 
	\begin{solution}
		\emph{Application.}
		\begin{verbatim}
		split [] = []
		split [_] = []
		split (x:xs) = ([x],xs) : [(x:ls,rs) | (ls,rs) <- split xs]
		\end{verbatim}
	\end{solution}
	\part[5] With the help of \texttt{\small split}, define a function 
	\begin{center}
		\texttt{\small exprs~::~[Int] -> [Expr]}
	\end{center}
	which generates all possible expressions for a list of numbers. All of the numbers from the list should be used and in the order provided. For example, \texttt{\small exprs [1,2]} should evaluate to \texttt{\small [App Add (Val 1) (Val 2), App Sub (Val 1) (Val 2), App Mul (Val 1) (Val 2), App Div (Val 1) \linebreak (Val 2)]}. \emph{Hint:} for longer lists, expressions will have to be nested further. \droppoints
	\begin{solution}
		\emph{Application}
		\begin{verbatim}
		exprs [] = []
		exprs [n] = [Val n]
		exprs ns = [e | (ls,rs) <- split ns,
		                      l <- exprs ls,
		                      r <- exprs rs,
		                      e <- [App o l r | 
		                              o <- [Add, Sub, Mul, Div]]]
		\end{verbatim}
	\end{solution}
	\part[5] Define a function 
	\begin{center}
		\texttt{\small eval~::~Expr -> [Int]}
	\end{center}
	which evaluates an expression. For example, \texttt{\small eval (App Div (Val 4) (Val 3)} should evaluate to \texttt{\small [1]}. This function should evaluate to \texttt{\small []} if the expression cannot be evaluated (e.g. due to division by zero). \droppoints 
	\begin{solution}
		\emph{Application.}
		\begin{verbatim}
		eval (Val n) = [n]
		eval (App o l r) = do 
		  x <- eval l
		  y <- eval r 
		  [apply o x y]
		
		apply Add x y = x + y
		apply Sub x y = x - y
		apply Mul x y = x * y
		apply Div x y = x `div` y
		\end{verbatim}
	\end{solution}
	\part[4] Define a function 
	\begin{center}
		\texttt{\small permutations~::~Eq a => [a] -> [[a]]}
	\end{center}
	which calculates all permutations of a list. For example, \texttt{\small permutations \linebreak{} [1,2,3]} should evaluate to \texttt{\small [[1,2,3],[2,1,3],[2,3,1],[1,3,2],\linebreak{}[3,1,2],[3,2,1]]}. \droppoints 
	\begin{solution}
		\emph{Bookwork.}
		\begin{verbatim}
		permutations :: Eq a => [a] -> [[a]]
		permutations [] = [[]]
		permutations xs = [x : ys | x <- xs, 
		                            ys <- permutations (delete x xs)]
		\end{verbatim}
	\end{solution}
	
	\part[3] With the help of \texttt{\small permutations}, define a function 
	\begin{center}
		\texttt{\small choices~::~[a] -> [[a]]}
	\end{center}
	which generates all permutations of subsequences of a list. For example, \texttt{\small choices [1,2,3]} should evaluate to \texttt{\small [[], [3], [2], [2, 3], [3, 2], \linebreak{} [1], [1, 3], [3, 1], [1, 2], [2, 1], [1, 2, 3], [2, 1, 3],}\\ \texttt{[2, 3, 1],[1, 3, 2], [3, 1, 2], [3, 2, 1]]}. \droppoints 
	\begin{solution}
		\emph{Application.}
		\begin{verbatim}
		choices = concat . map permutations . subsequences
		\end{verbatim}
	\end{solution}

	\pagebreak

	\part[4] With the help of \texttt{\small choices}, \texttt{\small exprs}, and \texttt{\small eval} define a function 
	\begin{center}
		\texttt{\small countdown~::~[Int] -> Int -> [Expr]}
	\end{center}
	which solves the countdown problem for a given list of numbers and a target number by returning a list of expressions which, when evaluated, result in the target number.  \droppoints 
	\begin{solution}
		\emph{Application.}
		\begin{verbatim}
		countdown ns n = [e | ns' <- choices ns,
	                            e <- exprs ns',
		                        eval e == [n]]
		\end{verbatim}
	\end{solution}
\end{parts}
 % done
		
%%% Question 3 - - - - - - - - - - - - - - - - - - - - - - - - - - - - - -
\question This question is about user-defined types and type classes.
\begin{parts}
	\part Consider the following data type:
	\begin{verbatim}
	data Layout a = Element a  
	              | Vertical (Layout a) (Layout a)
	              | Horizontal (Layout a) (Layout a)
	\end{verbatim}
	The intuition here is that a layout consists of elements which are composed vertically or horizontally.
	\begin{subparts}
		\subpart[1] What is the type of the \texttt{\small Element} constructor? \droppoints 
		
		\begin{solution}
			\emph{Comprehension.} \haskellIn{a -> Layout a}
		\end{solution}
	
		\subpart[2] What is the type of the \texttt{\small Vertical} constructor? \droppoints
		
		\begin{solution}
			\emph{Comprehension.} \haskellIn{Layout a -> Layout a -> Layout a}
		\end{solution}
	
		\subpart[1] What is the kind of the \texttt{\small Layout} type? \droppoints 
		
		\begin{solution}
			\emph{Comprehension.} \haskellIn{* -> *}
		\end{solution}
	
		\subpart[4] The \texttt{\small Layout} type is a functor. Define a suitable instance of the \texttt{\small Functor} type class for it. \droppoints 
		
		\begin{solution}
			\emph{Application.} 
			\begin{verbatim}
			instance Functor Layout where
			  fmap f (Element a) = Element (f a)
			  fmap f (Vertical x y) = Vertical (fmap f x) (fmap f y)
			  fmap f (Horizontal x y) = 
			    Horizontal (fmap f x) (fmap f y) 
			\end{verbatim}
		\end{solution}
	\end{subparts}

	\part Consider the following definition: 
	\begin{verbatim}
	example :: Layout (Sized Char)
	example = Vertical 
	  (Element (Sized (15,42) 'a'))
	  (Horizontal 
	    (Element (Sized (4,16) 'b')) 
	    (Element (Sized (23,8) 'c')))
	\end{verbatim}
	
	\begin{subparts}
		\subpart[2] Define a suitable data type \texttt{\small Sized} so that the above definition is valid. \droppoints
		
		\begin{solution}
			\emph{Application.}
			\begin{verbatim}
			data Sized a = Sized (Int,Int) a
			\end{verbatim}
		\end{solution}
	
		\subpart[4] Define a function 
		\begin{center}
			\texttt{\small width~::~Layout (Sized a) -> Int}
		\end{center}
		which calculates the width of a layout. For example, \texttt{\small width example} should evaluate to \texttt{\small 27}. The width of horizontally composed layouts is the sum of the widths of the sub-layouts and the width of vertically composed layout is the maximum width of the sub-layouts.  \droppoints
		
		\begin{solution}
			\emph{Application.} 
			\begin{verbatim}
				width :: Layout (Sized a) -> Int
				width (Element (Sized (x,y) a)) = x 
				width (Vertical a b) = max (width a) (width b)
				width (Horizontal a b) = width a + width b
			\end{verbatim}
		\end{solution}
	
		\subpart[4] Define a corresponding function
		\begin{center}
			\texttt{\small height~::~Layout (Sized a) -> Int}
		\end{center}
		which calculates the height of a layout. For example, \texttt{\small width example} should evaluate to \texttt{\small 58}. \droppoints
		
		\begin{solution}
			\emph{Application.} 
			\begin{verbatim}
				height :: Layout (Sized a) -> Int
				height (Element (Sized (x,y) a)) = y 
				height (Horizontal a b) = max (height a) (height b)
				height (Vertical a b) = height a + height b
			\end{verbatim}
		\end{solution} 
	
		\subpart[7] Suppose that we wish to render a given layout to a surface that uses a coordinate system where the origin is the top left corner and that we need to know the top left coordinate of each element in a layout for this purpose. Define a function 
		\begin{center}
			\texttt{\small pos~::~Layout (Sized a) -> Layout (Sized (Int,Int))}
		\end{center}
		which, for every element in the layout, calculates its absolute position and replaces its value with the calculated position. For example, \texttt{\small pos example} should evaluate to the following: \droppoints
		\begin{verbatim}
		Vertical (Element (Sized (15,42) (0,0)))
		         (Horizontal 
		           (Element (Sized (4,16) (0,42))) 
		           (Element (Sized (23,8) (4,42))))
		\end{verbatim}
	\end{subparts}
\end{parts}
 % done
		\allowdisplaybreaks
%%% Question 4 - - - - - - - - - - - - - - - - - - - - - - - - - - - - - -
\question This question is about equational reasoning.
\begin{parts}
	
	\part[4] Suppose that the \texttt{\small length} function is defined as:
	\begin{verbatim}
	length :: [a] -> Int 
	length []     = 0 
	length (x:xs) = 1 + length xs
	\end{verbatim}
	Consider the following well known property which states that the \texttt{\small length} of two combined lists is the same as the sum of the individual lists' lengths:
	\begin{center}
		\texttt{\small length (xs ++ ys) == length xs + length ys}
	\end{center}
	Prove this property. You may assume standard properties of arithmetic. \droppoints 
	
	\begin{solution}
		\emph{Application.}
		\begin{verbatim}
		length ([] ++ ys) 
		= length ys 
		= 0 + length ys
		= length [] + length ys
		
		length ((x:xs) ++ ys)
		= length (x : (xs ++ ys))
		= 1 + length (xs++ys)
		= 1 + (length xs + length ys)
		= length (x:xs) + length ys
		\end{verbatim}
	\end{solution}
	
	\part[10] For a list of length $n$, there are $2^n$ many subsequences:
	\begin{center}
		\texttt{\small length (subsequences xs) = 2\textasciicircum (length xs)}
	\end{center}
	With the help of the previous property, standard properties of arithmetic, and any additional lemma(s) you need, which must also be proved, prove this property.  \droppoints
	
	\begin{solution}
		\emph{Application.}
		We first need to prove a lemma (1+3 marks):
		\begin{center}
			\texttt{\small length xs == length (map f xs)}
		\end{center}
		The proof is by induction on \texttt{\small xs}:
		\begin{verbatim}
		length [] 
		= { unapplying map }
		length (map f [])
		
		length (x:xs)
		= { applying length }
		1 + length xs 
		= { induction hypothesis }
		1 + length (map f xs)
		= { unapplying length }
		length (f x : map f xs)
		= { unapplying map }
		length (map f (x:xs))
		\end{verbatim}
		2+4 marks
		\begin{verbatim}
		Proof by induction on xs.
		
		Base case: length (subsequences []) = 2^(length [])
		length (subsequences [])
		= { applying subsequences }
		length [[]]
		= { applying length }
		1 
		= { arithmetic }
		2^1
		= { unapplying length }
		2^(length [])
		
		Inductive step: 
		  length (subsequences (x:xs)) = 2^(length (x:xs))
		  
		length (subsequences (x:xs))
		= { applying subsequences }
		length (subsequences xs ++ map (x:) (subsequences xs))
		= { length property }
		length (subsequences xs) + 
		  length (map (x:) (subsequences xs))
		= { lemma }
		length (subsequences xs) + length (subsequences xs)
		= { arithmetic }
		2 * length (subsequences xs)
		= { induction hypothesis }
		2 * 2^(length xs)
		= { arithmetic }
		2^(1+length xs)
		= { unapplying length }
		2^(length (x:xs))
		\end{verbatim}
	\end{solution}
	
	\part Consider Haskell's \texttt{\small Monoid} type class:
	\begin{verbatim}
	class Monoid a where 
	    mempty  :: a
	    mappend :: a -> a -> a
	\end{verbatim}
	Functions form a monoid if their co-domain is also a monoid: 
	\begin{verbatim}
	instance Monoid b => Monoid (a -> b) where 
	   mempty = \x -> mempty 
	   mappend f g = \x -> mappend (f x) (g x)
	\end{verbatim}
	Prove that the monoid laws hold for this instance of the \texttt{\small Monoid} type class.
	\begin{subparts}
		\subpart[3] Left identity: $\mathit{mappend}~f~\mathit{mempty} = f$ \droppoints
		
		\begin{solution}
			\emph{Application.}
			\begin{verbatim}
			mappend f mempty
			= \x -> mappend (f x) (mempty x)
			= \x -> mappend (f x) mempty
			= \x -> f x
			f
			\end{verbatim}
		\end{solution}
		
		\subpart[3] Right identity: $\mathit{mappend}~\mathit{mempty}~f = f$ \droppoints
		
		\begin{solution}
			\emph{Application.}
			\begin{verbatim}
			mappend mempty f
			= \x -> mappend (mempty x) (f x)
			= \x -> mappend mempty (f x)
			= \x -> f x
			f
			\end{verbatim}
		\end{solution}
		
		\subpart[5] Associativity: $\mathit{mappend}~f~(\mathit{mappend}~g~h) = \mathit{mappend}~(\mathit{mappend}~f~g)~h$ \droppoints
		
		\begin{solution}
			\emph{Application.}
			\begin{verbatim}
			mappend f (mappend g h)
			= \x -> mappend (f x) ((mappend g h) x)
			= \x -> mappend (f x) ((\y -> mappend (g y) (h y)) x)
			= \x -> mappend (f x) (mappend (g x) (h x))
			= \x -> mappend (mappend (f x) (g x)) (h x)
			= \x -> mappend ((\y -> mappend (f y) (g y)) x) (h x)
			= \x -> mappend ((mappend f g) x) (h x)
			= mappend (mappend f g) h
			\end{verbatim}
		\end{solution}
	\end{subparts}
\end{parts}
 % done
		
%%% Question 5 - - - - - - - - - - - - - - - - - - - - - - - - - - - - - -
\question This question is about functors, applicative functors, and monads. Consider the following type: % or maybe pairs?
\begin{verbatim}
data Pair a b = MkP a b
\end{verbatim}
\begin{parts} 
	\part[5] Define a suitable instance of the \haskellIn{Functor} type class for the \haskellIn{Pair} type and prove that your instance obeys the functor laws. \droppoints 
	
	\begin{solution}
	Instance (2 marks):
	\begin{verbatim}
	instance Functor (Pair a) where 
	  fmap f (MkP a b) = MkP a (f b)
	\end{verbatim}
	Proofs (1 mark for identity, 2 marks for fusion):
	\begin{verbatim}
	fmap id (MkP a b)
	= MkP a (id b)
	= MkP a b
	= id (MkP a b)
	
	fmap (f.g) (MkP a b)
	= MkP a ((f.g) b)
	= MkP a (f (g b))
	= fmap f (MkP a (g b))
	= fmap f (fmap g (MkP a b))
	= (fmap f . fmap g) (MkP a b)
	\end{verbatim}
	\end{solution}
	
	\part[15] Define a suitable instance of the \haskellIn{Applicative} type class for the \haskellIn{Pair} type, adding suitable type class constraints if necessary, and prove that your instance obeys the applicative functor laws. \droppoints
	\begin{solution}
	Instance (4 marks):
	\begin{verbatim}
	instance Monoid a => Applicative (Pair a) where 
	  pure x = MkP mempty x 
	  (MkP a f) <*> (MkP a' x) = MkP (mappend a a') (f x)
	\end{verbatim}
	Proofs (2 marks for Identity, 2 marks for Homomorphism, 3 for Interchange, 4 for Composition):
	\begin{verbatim}
	Identity: pure id <*> v = v 
	pure id <*> (MkP a b) 
	= MkP mempty id <*> (MkP a b) 
	= MkP (mappend mempty a) (id b)
	= MkP a b 
	
	Homomorphism: pure f <*> pure x = pure (f x)
	pure f <*> pure x 
	= MkP mempty f <*> MkP mempty x 
	= MkP (mappend mempty mempty) (f x)
	= MkP mempty (f x)
	
	Interchange: u <*> pure y = pure ($ y) <*> u 
	(MkP a f) <*> pure y 
	= (MkP a f) <*> (MkP mempty y)
	= MkP (mappend a mempty) (f y)
	= MkP mempty (f y)
	= MkP (mappend mempty a) (f $ y)
	= MkP (mappend mempty a) (($ y) f)
	= (MkP mempty ($ y)) <*> (MkP a f)
	= pure ($ y) <*> (MkP a f)
	
	Composition: pure (.) <*> u <*> v <*> w = u <*> (v <*> w)
	pure (.) <*> (MkP a x) <*> (MkP b y) <*> (MkP c z)
	= MkP mempty (.) <*> (MkP a x) <*> (MkP b y) <*> (MkP c z)
	= MkP (mappend mempty a) ((.) x) <*> (MkP b y) <*> (MkP c z)
	= MkP a ((.) x) <*> (MkP b y) <*> (MkP c z)
	= MkP (mappend a b) ((.) x y) <*> (MkP c z)
	= MkP (mappend (mappend a b) c) ((.) x y) z)
	= MkP (mappend (mappend a b) c) (x (y z))
	= MkP (mappend a (mappend b c)) (x (y z))
	= (MkP a x) <*> (MkP (mappend b c) (y z))
	= (MkP a x) <*> ((MkP b y) <*> (MkP c z))
	\end{verbatim}
	\end{solution}
	
	\part[5] Define a suitable instance of the \haskellIn{Monad} type class for the \haskellIn{Pair} type, adding suitable type class constraints if necessary. Although your instance of the \haskellIn{Monad} type class should obey the monad laws, you \emph{do not} need to prove this explicitly. \droppoints
	
	\begin{solution}
		\emph{Application.}
		\begin{verbatim}
		instance Monoid a => Monad (Pair a) where 
		  (MkP a x) >>= f = let (MkP b y) = f x 
		                    in MkP (mappend a b) y
		\end{verbatim}
	\end{solution}
\end{parts}
 % done
		%%% Question 6 - - - - - - - - - - - - - - - - - - - - - - - - - - - - - -
\question This question is about type-level programming. You may assume that all of GHC's language extensions are available to you for this question.
\begin{parts}
	\part[4] You are given the following definition of a data type in Haskell:
	\begin{verbatim}
	data List a = Empty | Cons a (List a)
	\end{verbatim}
	With reference to this definition, explain what is meant by \emph{type promotion}. In your answer, you should state relevant types and kinds that result from the definition of \texttt{\small List}. \droppoints 
	\begin{solution}
		\emph{Comprehension. 1 mark for an explanation that this definition normally results in two data constructors. 1 mark for correctly explaining/stating the types of those constructors. 1 mark for an explanation that with type promotion this definition also results in one new kind and two new types. 1 mark for correctly explaining/stating the kinds of those types. } 
	\end{solution}
	\part[2] Define a closed type family \texttt{\small Head} of kind \texttt{\small List a -> a} which computes the head of a type-level list. \droppoints 
	\begin{solution}
		\emph{Application.} \begin{verbatim}
		type family Head (xs :: List a) where
		  Head (Cons x xs) = x
		\end{verbatim}
	\end{solution}
	\part[4] Define a closed type family \texttt{\small Append} of kind \texttt{\small List a -> List a -> List a} which appends two type-level lists. \droppoints
	\begin{solution}
		\emph{Application.} \begin{verbatim}
		type family Append (xs :: List a) (ys :: List a) where
		  Append Empty ys = ys
		  Append (Cons x xs) ys = Cons x (Append xs ys)
		\end{verbatim}
	\end{solution}
	\part[4] Consider the following definition of a polymorphic proxy type:
	\begin{verbatim}
	data Proxy (k :: a) = Proxy
	\end{verbatim}
	With reference to this type, explain what proxy types are used for\linebreak in Haskell. \droppoints 
	\begin{solution}
		\emph{Bookwork.} In Haskell, only types of kind \texttt{*} have values. Therefore, functions' domains and codomains must be types of kind \texttt{*}. Proxies allow us to work around this restriction to pass type-level values to functions. The \texttt{Proxy} type above has kind \texttt{a -> *} and, if applied to a type argument, therefore has kind \texttt{*} which makes values of this type suitable as arguments to functions. 
	\end{solution}
	\part[6] With the help of the \texttt{\small Proxy} type and the \texttt{\small List} kind, define a singleton type for type-level lists. \droppoints 
	\begin{solution}
		\emph{Application. 1 mark for head, 1 mark for first equation, 4 marks for second equation.} \begin{verbatim}
		data ListSingleton (xs :: List a) where
		  SEmpty :: ListSingleton Empty
		  SCons  :: Proxy x -> ListSingleton xs -> 
		            ListSingleton (Cons x xs)
		\end{verbatim}
	\end{solution}
	\part[5] Given a type-level list, we can construct a curried function type where the types contained in the type-level list represent the parameters. Define a closed type family \texttt{\small Fun} of kind \texttt{\small List * -> * -> *} so that, for example,\linebreak \texttt{\small Fun (Cons Int (Cons String Empty)) Bool} evaluates to \texttt{\small Int -> String -> Bool}. \droppoints 
	\begin{solution}
		\emph{Application.} 
		\begin{verbatim}
		type family Fun (f :: List *) r where
		  Fun Empty      r = r
		  Fun (Cons a f) r = a -> Fun f r
		\end{verbatim}
	\end{solution}
\end{parts}
 % done
	\end{questions}

	\pagebreak
	%\newpage

%\begin{center}
%	\vspace*{4cm}
%	\textbf{This page is intentionally left almost blank.}
%\end{center}

%\newpage

\begin{multicols}{2}\scriptsize 
	
	\textbf{\normalsize Appendix: Prelude}
	
	\begin{verbatim}
	class Eq a where
	  (==), (/=) :: a -> a -> Bool
	  x /= y = not (x == y)
	\end{verbatim}
	
	\begin{verbatim}
	class Eq a => Ord a where
	  (<), (<=), (>), (>=) :: a -> a -> Bool
	  min, max             :: a -> a -> Bool
	
	  min x y | x <= y    = x
	          | otherwise = y
	
	  max x y | x <= y    = y
	          | otherwise = x
	\end{verbatim}
	
	\begin{verbatim}
	class Enum a where
	  succ           :: a -> a 
	  pred           :: a -> a
	  toEnum         :: Int -> a
	  fromEnum       :: a -> Int 
	\end{verbatim}
	
	\begin{verbatim}
	class Bounded a where
	  minBound :: a 
	  maxBound :: a
	\end{verbatim}
	
	\begin{verbatim}
	class Num a where 
	  (+), (-), (*) :: a -> a -> a
	  negate        :: a -> a
	  abs           :: a -> a
	  signum        :: a -> a
	  fromInteger   :: Integer -> a
	\end{verbatim}
	
	\begin{verbatim}
	class Enum a => Integral a where 
	  quot      :: a -> a -> a
	  rem       :: a -> a -> a
	  div       :: a -> a -> a
	  mod       :: a -> a -> a
	  quotRem   :: a -> a -> (a, a)
	  divMod    :: a -> a -> (a, a)
	  toInteger :: a -> Integer
	\end{verbatim}
	
	\begin{verbatim}
	class Num a => Fractional a where
	  (/)          :: a -> a -> a
	  recip        :: a -> a
	  fromRational :: Rational -> a
	\end{verbatim}
	
	\begin{verbatim}
	data Int = ...
	  deriving ( Eq, Ord, Show, Read
	           , Num, Integral )
	\end{verbatim}
	
	\begin{verbatim}
	data Integer = ...
	  deriving ( Eq, Ord, Show, Read
	           , Num, Integral )
	\end{verbatim}
	
	\begin{verbatim}
	data Float = ...
	  deriving ( Eq, Ord, Show, Read
	           , Num, Fractional )
	\end{verbatim}
	
	\begin{verbatim}
	data Double = ...
	  deriving ( Eq, Ord, Show, Read
	           , Num, Fractional )
	\end{verbatim}
	
	\begin{verbatim}
	even :: Integral a => a -> Bool 
	even n = n `mod` 2 == 0
	\end{verbatim}
	
	\begin{verbatim}
	odd :: Integral a => a -> Bool 
	odd = not . even
	\end{verbatim}
	
	\begin{verbatim}
	class Show a where
	  show :: a -> String
	\end{verbatim}
	
	\begin{verbatim}
	class Read a where 
	  read :: String -> a
	\end{verbatim}
	
	\begin{verbatim}
	class Foldable t where
	  foldr   :: (a -> b -> b) -> b -> t a -> b
	  foldl   :: (b -> a -> b) -> b -> t a -> b
	  foldr1  :: (a -> a -> a) -> t a -> a
	  foldl1  :: (a -> a -> a) -> t a -> a
	  
	  null :: t a -> Bool 
	  null = foldr (\_ _ -> False) True
	  
	  length :: t a -> Int
	  length = foldr (\x r -> 1 + r) 0
	  
	  elem :: Eq a -> a -> t a -> Bool 
	  elem x = foldr (\y r -> x==y || r) False
	  
	  maximum :: Ord a => t a -> a 
	  maximum = foldl1 max
	  
	  minimum :: Ord a => t a -> a 
	  minimum = foldl1 min
	  
	  sum :: Num a => t a -> a 
	  sum = foldl (+) 0
	  
	  product :: Num a => t a -> a
	  product = foldl (*) 1
	\end{verbatim}
	
	\begin{verbatim}
	(.) :: (b -> c) -> (a -> b) -> a -> c
	(.) f g x = f (g x)
	\end{verbatim}
	
	\begin{verbatim}
	id :: a -> a
	id x = x
	\end{verbatim}
	
	\begin{verbatim}
	const :: a -> b -> a
	const x _ = x
	\end{verbatim}
	
	\begin{verbatim}
	($!) :: (a -> b) -> a -> b
	f $! x = ...
	\end{verbatim}
	
	\subsection*{Functors}
	
	\begin{verbatim}
	class Functor f where 
	  fmap :: (a -> b) -> f a -> f b
	\end{verbatim}
	\begin{displaymath}
	\begin{array}{lrcl}
	\textbf{Identity} & \mathit{fmap}~\mathit{id} & = &  \mathit{id} \\
	\textbf{Fusion} & \mathit{fmap}~(f \circ g) & = & \mathit{fmap}~f \circ \mathit{fmap}~g
	\end{array}
	\end{displaymath}
	
	\subsection*{Applicatives}
	
	\begin{verbatim}
	class Functor f => Applicative f where 
	  pure  :: a -> f a
	  (<*>) :: f (a -> b) -> f a -> f b
	\end{verbatim}
	\begin{displaymath}
	\begin{array}{lr}
	\textbf{Identity} & \mathit{pure}~\mathit{id} <\!\!*\!\!> v = v \\ 
	\textbf{Homomorphism} & \mathit{pure}~f <\!\!*\!\!> \mathit{pure}~x \\& = \mathit{pure}~(f~x) \\
	\textbf{Interchange} & u <\!\!*\!\!> \mathit{pure}~y \\
	& = \mathit{pure}~(\$~y) <\!\!*\!\!> u \\
	\textbf{Composition} & \mathit{pure}~(\circ) <\!\!*\!\!> u <\!\!*\!\!> v <\!\!*\!\!> w  \\
	& = u <\!\!*\!\!> (v <\!\!*\!\!> w)
	\end{array}
	\end{displaymath}
	
	\subsection*{Monads}
	
	\begin{verbatim}
	class Applicative m => Monad m where 
	  return :: a -> m a
	  return = pure
	
	  (>>=) :: m a -> (a -> m b) -> m b
	\end{verbatim}
	
	\begin{displaymath}
	\begin{array}{lrcl}
	\textbf{Left identity} & \mathit{return}~a \bind f & = & f~a \\
	\textbf{Right identity} & m \bind \mathit{return} & = & m \\
	\textbf{Associativity} & (m \bind f) \bind g & = & \\ \multicolumn{2}{r}{ m \bind (\textbackslash x \to f~x \bind g)}
	\end{array}
	\end{displaymath}
	
	\subsection*{Booleans}
	
	\begin{verbatim}
	data Bool = True | False
	  deriving ( Bounded, Enum, Eq, Ord
	           , Read, Show )
	\end{verbatim}
	
	\begin{verbatim}
	not :: Bool -> Bool
	not True  = False
	not False = True
	\end{verbatim}
	
	\begin{verbatim}
	(&&) :: Bool -> Bool -> Bool
	True && True = True 
	_    && _    = False
	\end{verbatim}
	
	\begin{verbatim}
	(||) :: Bool -> Bool -> Bool
	False || False = False 
	_     || _     = True
	\end{verbatim}
	
	\begin{verbatim}
	and :: Foldable t => t Bool -> Bool 
	and = foldr (&&) True
	\end{verbatim}
	
	\begin{verbatim}
	or :: Foldable t => t Bool -> Bool 
	or = foldr (||) False
	\end{verbatim}
	
	\begin{verbatim}
	all :: Foldable t => 
	       (a -> Bool) -> t a -> Bool
	all p = and . foldr (\x xs -> p x : xs) []
	\end{verbatim}
	
	\begin{verbatim}
	any :: Foldable t => 
	       (a -> Bool) -> t a -> Bool
	any p = or . foldr (\x xs -> p x : xs) []
	\end{verbatim}
	
	\begin{verbatim}
	otherwise :: Bool
	otherwise = True
	\end{verbatim}
	
	\subsection*{Characters} 
	
	\begin{verbatim}
	data Char = ...
	\end{verbatim}
	
	\begin{verbatim}
	type String = [Char]
	\end{verbatim}
	
	\begin{verbatim}
	isLower :: Char -> Bool 
	isLower c = c >= 'a' && c <= 'z'
	\end{verbatim}
	
	\begin{verbatim}
	isUpper :: Char -> Bool 
	isUpper c = c >= 'A' && c <= 'Z'
	\end{verbatim}
	
	\begin{verbatim}
	isAlpha :: Char -> Bool 
	isAlpha c = isLower c || isUpper c
	\end{verbatim}
	
	\begin{verbatim}
	isDigit :: Char -> Bool 
	isDigit c = c >= '0' && c <= '9'
	\end{verbatim}
	
	\begin{verbatim}
	isAlphaNum :: Char -> Bool 
	isAlphaNum c = isAlpha c || isDigit c
	\end{verbatim}
	
	\begin{verbatim}
	isSpace :: Char -> Bool 
	isSpace c = c `elem` " \t\n"
	\end{verbatim}
	
	\begin{verbatim}
	ord :: Char -> Int 
	ord c = ...
	\end{verbatim}
	
	\begin{verbatim}
	chr :: Int -> Char 
	chr n = ...
	\end{verbatim}
	
	\begin{verbatim}
	digitToInt :: Char -> Int 
	digitToInt c | isDigit c = ord c - ord '0'
	\end{verbatim}
	
	\begin{verbatim}
	intToDigit :: Int -> Char 
	intToDigit n 
	  | n >= 0 && n <= 9 = chr (ord '0' + n)
	\end{verbatim}
	
	\begin{verbatim}
	toLower :: Char -> Char 
	toLower c 
	  | isUpper c =
	      chr (ord c - ord 'A' + ord 'a')
	  | otherwise = c
	\end{verbatim}
	
	\begin{verbatim}
	toUpper :: Char -> Char 
	toUpper c 
	  | isLower c =
	      chr (ord c - ord 'a' + ord 'A')
	  | otherwise = c
	\end{verbatim}
	
	\subsection*{Lists}
	
	\begin{verbatim}
	data [a] = [] | (:) a [a]
	  deriving (Eq, Ord, Show, Read)
	\end{verbatim}
	
	\begin{verbatim}
	instance Functor [] where 
	  fmap = map
	\end{verbatim}
	
	\begin{verbatim}
	instance Applicative [] where
	  pure x = [x]
	  
	  fs <*> xs = [f x | f <- fs, x <- xs]
	\end{verbatim}
	
	\begin{verbatim}
	instance Monad [] where 
	  xs >>= f = [y | x <- xs, y <- f x]
	\end{verbatim}
	
	\begin{verbatim}
	instance Foldable [] where 
	  foldr _ v []     = v 
	  foldr f v (x:xs) = f x (foldr f v xs)
	  
	  foldr1 _ [x]    = x 
	  foldr1 f (x:xs) = f x (foldr1 f xs)
	  
	  foldl _ v []     = v 
	  foldl f v (x:xs) = foldl f (f v x) xs
	  
	  foldl1 f (x:xs) = foldl f x xs
	\end{verbatim}
	
	\begin{verbatim}
	head :: [a] -> a 
	head (x:xs) = x
	
	tail :: [a] -> [a]
	tail (x:xs) = xs
	\end{verbatim}
	
	\begin{verbatim}
	last :: [a] -> a
	last [x]    = x
	last (x:xs) = last xs
	\end{verbatim}
	
	\begin{verbatim}
	init :: [a] -> [a]
	init [_]    = []
	init (x:xs) = x : init xs
	\end{verbatim}
	
	\begin{verbatim}
	map :: (a -> b) -> [a] -> [b]
	map f []     = []
	map f (x:xs) = f x : map f xs
	\end{verbatim}
	
	\begin{verbatim}
	filter :: (a -> Bool) -> [a] -> [a]
	filter p [] = []
	filter p (x:xs)
	  | p x       = x : filter p xs
	  | otherwise = filter p xs
	\end{verbatim}
	
	\begin{verbatim}
	lookup :: Eq k => k -> [(k,v)] -> Maybe v
	lookup x [] = Nothing
	lookup x ((y,v):ys)
	  | x == y    = Just v
	  | otherwise = lookup x ys
	\end{verbatim}
	
	\begin{verbatim}
	(!!) :: [a] -> Int -> a
	(x:xs) !! 0 = x
	(x:xs) !! n = xs !! (n-1)
	\end{verbatim}
	
	\begin{verbatim}
	take :: Int -> [a] -> [a]
	take 0 _      = []
	take n []     = []
	take n (x:xs) = x : take (n-1) xs
	\end{verbatim}
	
	\begin{verbatim}
	drop :: Int -> [a] -> [a]
	drop 0 xs     = xs 
	drop n []     = []
	drop n (x:xs) = drop (n-1) xs
	\end{verbatim}
	
	\begin{verbatim}
	takeWhile :: (a -> Bool) -> [a] -> [a]
	takeWhile _ [] = []
	takeWhile p (x:xs) 
	  | p x       = x : takeWhile p xs
	  | otherwise = []
	\end{verbatim}
	
	\begin{verbatim}
	dropWhile :: (a -> Bool) -> [a] -> [a]
	dropWhile _ [] = []
	dropWhile p (x:xs) 
	  | p x       = dropWhile p xs
	  | otherwise = x : xs
	\end{verbatim}
	
	\begin{verbatim}
	splitAt :: Int -> [a] -> ([a], [a])
	splitAt n xs = (take n xs, drop n xs)
	\end{verbatim}
	
	\begin{verbatim}
	span :: (a -> Bool) -> [a] -> ([a], [a])
	span p xs = 
	  (takeWhile p xs, dropWhile p xs)
	\end{verbatim}
	
	\begin{verbatim}
	repeat :: a -> [a]
	repeat x = xs where xs = x : xs
	\end{verbatim}
	
	\begin{verbatim}
	replicate :: Int -> a -> [a]
	replicate n = take n . repeat
	\end{verbatim}
	
	\begin{verbatim}
	iterate :: (a -> a) -> a -> [a]
	iterate f x = x : iterate f (f x)
	\end{verbatim}
	
	\begin{verbatim}
	zip :: [a] -> [b] -> [(a,b)]
	zip []     _      = []
	zip _      []     = []
	zip (x:xs) (y:ys) = (x,y) : zip xs ys
	\end{verbatim}
	
	\begin{verbatim}
	(++) :: [a] -> [a] -> [a]
	[]     ++ ys = ys
	(x:xs) ++ ys = x : (xs ++ ys)
	\end{verbatim}
	
	\begin{verbatim}
	concat :: [[a]] -> [a]
	concat = foldr (++) []
	\end{verbatim}
	
	\begin{verbatim}
	reverse :: [a] -> [a]
	reverse = foldl (\xs x -> x : xs) []
	\end{verbatim}
	
	\begin{verbatim}
	subsequences :: [a] -> [[a]]
	subsequences []     = [[]]
	subsequences (x:xs) = ys ++ map (x:) ys
	  where ys = subsequences xs
	\end{verbatim}
	
	\begin{verbatim}
	nub :: Eq a => [a] -> [a]
	nub []     = []
	nub (x:xs) = x : nub (filter (/= x) xs)
	\end{verbatim}
	
	\begin{verbatim}
	delete :: Eq a => a -> [a] -> [a]
	delete _ []     = []
	delete x (y:ys) 
	  | x == y    = ys
	  | otherwise = y : delete x ys
	\end{verbatim}
	
	\subsection*{Maybe}
	
	\begin{verbatim}
	data Maybe a = Nothing | Just a
	  deriving (Eq, Ord, Read, Show)
	\end{verbatim}
	
	\begin{verbatim}
	instance Functor Maybe where
	  fmap f Nothing  = Nothing
	  fmap f (Just x) = Just (f x)
	\end{verbatim}
	
	\begin{verbatim}
	instance Applicative Maybe where
	  pure x = Just x
	  
	  Nothing  <*> _ = Nothing
	  (Just f) <*> y = fmap f y
	\end{verbatim}
	
	\begin{verbatim}
	instance Monad Maybe where
	  Nothing  >>= f = Nothing
	  (Just x) >>= f = f x
	\end{verbatim}
	
	\subsection*{Either}
	
	\begin{verbatim}
	data Either a b = Left a | Right b
	\end{verbatim}
	
	\subsection*{Tuples}
	
	All tuples are instances of \texttt{Eq}, \texttt{Ord}, \texttt{Show}, \texttt{Read}.
	
	\begin{verbatim}
	fst :: (a, b) -> a
	fst (x,y) = x
	\end{verbatim}
	
	\begin{verbatim}
	snd :: (a, b) -> b
	snd (x,y) = y
	\end{verbatim}
	
	\begin{verbatim}
	curry :: ((a, b) -> c) -> a -> b -> c 
	curry f x y = f (x, y)
	\end{verbatim}
	
	\begin{verbatim}
	uncurry :: (a -> b -> c) -> (a, b) -> c
	uncurry f (x,y) = f x y
	\end{verbatim}
	
	\subsection*{IO}
	
	\begin{verbatim}
	data IO a = ...
	
	instance Functor IO where ...
	instance Applicative IO where ...
	instance Monad IO where ...
	\end{verbatim}
	
	\begin{verbatim}
	getChar :: IO Char
	getChar = ...
	
	getLine :: IO String
	getLine = ...
	
	putChar :: Char -> IO ()
	putChar c = ...
	
	putStr :: String -> IO ()
	putStr []     = return ()
	putStr (x:xs) = putChar x >> putStr xs
	
	putStrLn :: String -> IO ()
	putStrLn xs = putStr xs >> putChar '\n'
	
	print :: Show a => a -> IO ()
	print = putStrLn . show
	\end{verbatim}
	
	\subsection*{Type-level programming}
	
	The kind of types is denoted as \texttt{*}. 
	
	\begin{verbatim}
	data Nat = Zero | Succ Nat
	\end{verbatim}
\end{multicols}

\end{document}