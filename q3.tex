
%%% Question 3 - - - - - - - - - - - - - - - - - - - - - - - - - - - - - -
\question This question is about user-defined types and type classes.
\begin{parts}
	\part Consider the following data type:
	\begin{verbatim}
	data Layout a = Element a  
	              | Vertical (Layout a) (Layout a)
	              | Horizontal (Layout a) (Layout a)
	\end{verbatim}
	The intuition here is that a layout consists of elements which are composed vertically or horizontally.
	\begin{subparts}
		\subpart[1] What is the type of the \texttt{\small Element} constructor? \droppoints 
		
		\begin{solution}
			\emph{Comprehension.} \haskellIn{a -> Layout a}
		\end{solution}
	
		\subpart[2] What is the type of the \texttt{\small Vertical} constructor? \droppoints
		
		\begin{solution}
			\emph{Comprehension.} \haskellIn{Layout a -> Layout a -> Layout a}
		\end{solution}
	
		\subpart[1] What is the kind of the \texttt{\small Layout} type? \droppoints 
		
		\begin{solution}
			\emph{Comprehension.} \haskellIn{* -> *}
		\end{solution}
	
		\subpart[4] The \texttt{\small Layout} type is a functor. Define a suitable instance of the \texttt{\small Functor} type class for it. \droppoints 
		
		\begin{solution}
			\emph{Application.} 
			\begin{verbatim}
			instance Functor Layout where
			  fmap f (Element a) = Element (f a)
			  fmap f (Vertical x y) = Vertical (fmap f x) (fmap f y)
			  fmap f (Horizontal x y) = 
			    Horizontal (fmap f x) (fmap f y) 
			\end{verbatim}
		\end{solution}
	\end{subparts}

	\part Consider the following definition: 
	\begin{verbatim}
	example :: Layout (Sized Char)
	example = Vertical 
	  (Element (Sized (15,42) 'a'))
	  (Horizontal 
	    (Element (Sized (4,16) 'b')) 
	    (Element (Sized (23,8) 'c')))
	\end{verbatim}
	
	\begin{subparts}
		\subpart[2] Define a suitable data type \texttt{\small Sized} so that the above definition is valid. \droppoints
		
		\begin{solution}
			\emph{Application.}
			\begin{verbatim}
			data Sized a = Sized (Int,Int) a
			\end{verbatim}
		\end{solution}
	
		\subpart[4] Define a function 
		\begin{center}
			\texttt{\small width~::~Layout (Sized a) -> Int}
		\end{center}
		which calculates the width of a layout. For example, \texttt{\small width example} should evaluate to \texttt{\small 27}. The width of horizontally composed layouts is the sum of the widths of the sub-layouts and the width of vertically composed layout is the maximum width of the sub-layouts.  \droppoints
		
		\begin{solution}
			\emph{Application.} 
			\begin{verbatim}
				width :: Layout (Sized a) -> Int
				width (Element (Sized (x,y) a)) = x 
				width (Vertical a b) = max (width a) (width b)
				width (Horizontal a b) = width a + width b
			\end{verbatim}
		\end{solution}
	
		\subpart[4] Define a corresponding function
		\begin{center}
			\texttt{\small height~::~Layout (Sized a) -> Int}
		\end{center}
		which calculates the height of a layout. For example, \texttt{\small width example} should evaluate to \texttt{\small 58}. \droppoints
		
		\begin{solution}
			\emph{Application.} 
			\begin{verbatim}
				height :: Layout (Sized a) -> Int
				height (Element (Sized (x,y) a)) = y 
				height (Horizontal a b) = max (height a) (height b)
				height (Vertical a b) = height a + height b
			\end{verbatim}
		\end{solution} 
	
		\subpart[7] Suppose that we wish to render a given layout to a surface that uses a coordinate system where the origin is the top left corner and that we need to know the top left coordinate of each element in a layout for this purpose. Define a function 
		\begin{center}
			\texttt{\small pos~::~Layout (Sized a) -> Layout (Sized (Int,Int))}
		\end{center}
		which, for every element in the layout, calculates its absolute position and replaces its value with the calculated position. For example, \texttt{\small pos example} should evaluate to the following: \droppoints
		\begin{verbatim}
		Vertical (Element (Sized (15,42) (0,0)))
		         (Horizontal 
		           (Element (Sized (4,16) (0,42))) 
		           (Element (Sized (23,8) (4,42))))
		\end{verbatim}
	\end{subparts}
\end{parts}
