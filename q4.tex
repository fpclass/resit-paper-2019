\allowdisplaybreaks
%%% Question 4 - - - - - - - - - - - - - - - - - - - - - - - - - - - - - -
\question This question is about equational reasoning.
\begin{parts}
	
	\part[4] Suppose that the \texttt{\small length} function is defined as:
	\begin{verbatim}
	length :: [a] -> Int 
	length []     = 0 
	length (x:xs) = 1 + length xs
	\end{verbatim}
	Consider the following well known property which states that the \texttt{\small length} of two combined lists is the same as the sum of the individual lists' lengths:
	\begin{center}
		\texttt{\small length (xs ++ ys) == length xs + length ys}
	\end{center}
	Prove this property. You may assume standard properties of arithmetic. \droppoints 
	
	\begin{solution}
		\emph{Application.}
		\begin{verbatim}
		length ([] ++ ys) 
		= length ys 
		= 0 + length ys
		= length [] + length ys
		
		length ((x:xs) ++ ys)
		= length (x : (xs ++ ys))
		= 1 + length (xs++ys)
		= 1 + (length xs + length ys)
		= length (x:xs) + length ys
		\end{verbatim}
	\end{solution}
	
	\part[10] For a list of length $n$, there are $2^n$ many subsequences:
	\begin{center}
		\texttt{\small length (subsequences xs) = 2\textasciicircum (length xs)}
	\end{center}
	With the help of the previous property, standard properties of arithmetic, and any additional lemma(s) you need, which must also be proved, prove this property.  \droppoints
	
	\begin{solution}
		\emph{Application.}
		We first need to prove a lemma (1+3 marks):
		\begin{center}
			\texttt{\small length xs == length (map f xs)}
		\end{center}
		The proof is by induction on \texttt{\small xs}:
		\begin{verbatim}
		length [] 
		= { unapplying map }
		length (map f [])
		
		length (x:xs)
		= { applying length }
		1 + length xs 
		= { induction hypothesis }
		1 + length (map f xs)
		= { unapplying length }
		length (f x : map f xs)
		= { unapplying map }
		length (map f (x:xs))
		\end{verbatim}
		2+4 marks
		\begin{verbatim}
		Proof by induction on xs.
		
		Base case: length (subsequences []) = 2^(length [])
		length (subsequences [])
		= { applying subsequences }
		length [[]]
		= { applying length }
		1 
		= { arithmetic }
		2^1
		= { unapplying length }
		2^(length [])
		
		Inductive step: 
		  length (subsequences (x:xs)) = 2^(length (x:xs))
		  
		length (subsequences (x:xs))
		= { applying subsequences }
		length (subsequences xs ++ map (x:) (subsequences xs))
		= { length property }
		length (subsequences xs) + 
		  length (map (x:) (subsequences xs))
		= { lemma }
		length (subsequences xs) + length (subsequences xs)
		= { arithmetic }
		2 * length (subsequences xs)
		= { induction hypothesis }
		2 * 2^(length xs)
		= { arithmetic }
		2^(1+length xs)
		= { unapplying length }
		2^(length (x:xs))
		\end{verbatim}
	\end{solution}
	
	\part Consider Haskell's \texttt{\small Monoid} type class:
	\begin{verbatim}
	class Monoid a where 
	    mempty  :: a
	    mappend :: a -> a -> a
	\end{verbatim}
	Functions form a monoid if their co-domain is also a monoid: 
	\begin{verbatim}
	instance Monoid b => Monoid (a -> b) where 
	   mempty = \x -> mempty 
	   mappend f g = \x -> mappend (f x) (g x)
	\end{verbatim}
	Prove that the monoid laws hold for this instance of the \texttt{\small Monoid} type class.
	\begin{subparts}
		\subpart[3] Left identity: $\mathit{mappend}~f~\mathit{mempty} = f$ \droppoints
		
		\begin{solution}
			\emph{Application.}
			\begin{verbatim}
			mappend f mempty
			= \x -> mappend (f x) (mempty x)
			= \x -> mappend (f x) mempty
			= \x -> f x
			f
			\end{verbatim}
		\end{solution}
		
		\subpart[3] Right identity: $\mathit{mappend}~\mathit{mempty}~f = f$ \droppoints
		
		\begin{solution}
			\emph{Application.}
			\begin{verbatim}
			mappend mempty f
			= \x -> mappend (mempty x) (f x)
			= \x -> mappend mempty (f x)
			= \x -> f x
			f
			\end{verbatim}
		\end{solution}
		
		\subpart[5] Associativity: $\mathit{mappend}~f~(\mathit{mappend}~g~h) = \mathit{mappend}~(\mathit{mappend}~f~g)~h$ \droppoints
		
		\begin{solution}
			\emph{Application.}
			\begin{verbatim}
			mappend f (mappend g h)
			= \x -> mappend (f x) ((mappend g h) x)
			= \x -> mappend (f x) ((\y -> mappend (g y) (h y)) x)
			= \x -> mappend (f x) (mappend (g x) (h x))
			= \x -> mappend (mappend (f x) (g x)) (h x)
			= \x -> mappend ((\y -> mappend (f y) (g y)) x) (h x)
			= \x -> mappend ((mappend f g) x) (h x)
			= mappend (mappend f g) h
			\end{verbatim}
		\end{solution}
	\end{subparts}
\end{parts}
