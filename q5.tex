
%%% Question 5 - - - - - - - - - - - - - - - - - - - - - - - - - - - - - -
\question This question is about functors, applicative functors, and monads. Consider the following type: % or maybe pairs?
\begin{verbatim}
data Pair a b = MkP a b
\end{verbatim}
\begin{parts} 
	\part[5] Define a suitable instance of the \haskellIn{Functor} type class for the \haskellIn{Pair} type and prove that your instance obeys the functor laws. \droppoints 
	
	\begin{solution}
	Instance (2 marks):
	\begin{verbatim}
	instance Functor (Pair a) where 
	  fmap f (MkP a b) = MkP a (f b)
	\end{verbatim}
	Proofs (1 mark for identity, 2 marks for fusion):
	\begin{verbatim}
	fmap id (MkP a b)
	= MkP a (id b)
	= MkP a b
	= id (MkP a b)
	
	fmap (f.g) (MkP a b)
	= MkP a ((f.g) b)
	= MkP a (f (g b))
	= fmap f (MkP a (g b))
	= fmap f (fmap g (MkP a b))
	= (fmap f . fmap g) (MkP a b)
	\end{verbatim}
	\end{solution}
	
	\part[15] Define a suitable instance of the \haskellIn{Applicative} type class for the \haskellIn{Pair} type, adding suitable type class constraints if necessary, and prove that your instance obeys the applicative functor laws. \droppoints
	\begin{solution}
	Instance (4 marks):
	\begin{verbatim}
	instance Monoid a => Applicative (Pair a) where 
	  pure x = MkP mempty x 
	  (MkP a f) <*> (MkP a' x) = MkP (mappend a a') (f x)
	\end{verbatim}
	Proofs (2 marks for Identity, 2 marks for Homomorphism, 3 for Interchange, 4 for Composition):
	\begin{verbatim}
	Identity: pure id <*> v = v 
	pure id <*> (MkP a b) 
	= MkP mempty id <*> (MkP a b) 
	= MkP (mappend mempty a) (id b)
	= MkP a b 
	
	Homomorphism: pure f <*> pure x = pure (f x)
	pure f <*> pure x 
	= MkP mempty f <*> MkP mempty x 
	= MkP (mappend mempty mempty) (f x)
	= MkP mempty (f x)
	
	Interchange: u <*> pure y = pure ($ y) <*> u 
	(MkP a f) <*> pure y 
	= (MkP a f) <*> (MkP mempty y)
	= MkP (mappend a mempty) (f y)
	= MkP mempty (f y)
	= MkP (mappend mempty a) (f $ y)
	= MkP (mappend mempty a) (($ y) f)
	= (MkP mempty ($ y)) <*> (MkP a f)
	= pure ($ y) <*> (MkP a f)
	
	Composition: pure (.) <*> u <*> v <*> w = u <*> (v <*> w)
	pure (.) <*> (MkP a x) <*> (MkP b y) <*> (MkP c z)
	= MkP mempty (.) <*> (MkP a x) <*> (MkP b y) <*> (MkP c z)
	= MkP (mappend mempty a) ((.) x) <*> (MkP b y) <*> (MkP c z)
	= MkP a ((.) x) <*> (MkP b y) <*> (MkP c z)
	= MkP (mappend a b) ((.) x y) <*> (MkP c z)
	= MkP (mappend (mappend a b) c) ((.) x y) z)
	= MkP (mappend (mappend a b) c) (x (y z))
	= MkP (mappend a (mappend b c)) (x (y z))
	= (MkP a x) <*> (MkP (mappend b c) (y z))
	= (MkP a x) <*> ((MkP b y) <*> (MkP c z))
	\end{verbatim}
	\end{solution}
	
	\part[5] Define a suitable instance of the \haskellIn{Monad} type class for the \haskellIn{Pair} type, adding suitable type class constraints if necessary. Although your instance of the \haskellIn{Monad} type class should obey the monad laws, you \emph{do not} need to prove this explicitly. \droppoints
	
	\begin{solution}
		\emph{Application.}
		\begin{verbatim}
		instance Monoid a => Monad (Pair a) where 
		  (MkP a x) >>= f = let (MkP b y) = f x 
		                    in MkP (mappend a b) y
		\end{verbatim}
	\end{solution}
\end{parts}
